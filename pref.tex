\section*{Advertencia}

La Real Academia Española aprobó, en 1952, con carácter optativo, las Nuevas Normas de Prosodia y Ortografía. Posteriormente, como resultado del II Congreso de Academias de la Lengua Española, y teniendo en cuenta el informe previo de los países hispanoamericanos,introdujo en las mismas ligeras modificaciones y sancionó sus definitivas prescripciones, declarándolas de aplicación obligatoria a partir de enero de 1959.

\par Las regla didácticas de este libro recogen , en su totalidad, las NUEVAS NORMAS definitivas su texto está escrito observándolas estrictamente.

\chapter*{Prefacio}
En esta trigésima segunda edición de mi \textsc{Análisis Gramatical}, algunos conceptos han sido objeto de rectificación de reforma en consonancia con las "Nuevas Normas \textit{definitivas} definitivas", dictadas por la Real Academia.

\par Con apego siempre a la teoría oficial, que es a la que han de ajustarse todos los tribunales de exámenes;  de esta forma, doy cumplimiento a la misión que la Academia atribuye a los tratadistas de \textit{analizar}, \textit{explicar}, y \textit{difundir sus doctrinas con fines especialmente pedagógicos}.

\par No por eso dejo de hacer aclaraciones sobre ciertos detalles que la Academia no menciona y sobre otros que producecn confusión (véase Nota en la página siguiente).

\par Siempre que hago mención de la \textit{Gramática de la Academia}, me refiero a la edición reformada de 1931, que es la última publicada.
